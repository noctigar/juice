%% Generated by Sphinx.
\def\sphinxdocclass{jsbook}
\documentclass[letterpaper,10pt,dvipdfmx]{sphinxmanual}
\ifdefined\pdfpxdimen
   \let\sphinxpxdimen\pdfpxdimen\else\newdimen\sphinxpxdimen
\fi \sphinxpxdimen=.75bp\relax



\usepackage{cmap}
\usepackage[T1]{fontenc}
\usepackage{amsmath,amssymb,amstext}

\usepackage{times}

\usepackage{longtable}
\usepackage{sphinx}

\usepackage[dvipdfm]{geometry}
\usepackage{multirow}
\usepackage{eqparbox}

% Include hyperref last.
\usepackage{hyperref}
% Fix anchor placement for figures with captions.
\usepackage{hypcap}% it must be loaded after hyperref.
% Set up styles of URL: it should be placed after hyperref.
\urlstyle{same}
\renewcommand{\contentsname}{ユーザー側の操作API}

\renewcommand{\figurename}{図}
\renewcommand{\tablename}{TABLE}
\renewcommand{\literalblockname}{LIST}

\def\pageautorefname{ページ}

\setcounter{tocdepth}{1}



\title{juice\_sale\_system Documentation}
\date{5月 23, 2017}
\release{0.5}
\author{hatashi,yanai}
\newcommand{\sphinxlogo}{}
\renewcommand{\releasename}{リリース}
\makeindex

\begin{document}

\maketitle
\sphinxtableofcontents
\phantomsection\label{\detokenize{index::doc}}



\chapter{商品一覧API}
\label{\detokenize{user/index:api}}\label{\detokenize{user/index::doc}}\label{\detokenize{user/index:welcome-to-juice-sale-system-s-documentation}}

\section{内容}
\label{\detokenize{user/index:id1}}\begin{enumerate}
\item {} 
商品の一覧を取得する。

\end{enumerate}


\subsection{リクエスト}
\label{\detokenize{user/index:id2}}
/item


\subsection{返ってくる値}
\label{\detokenize{user/index:id3}}\begin{itemize}
\item {} 
商品名

\item {} 
あったかい or つめたい

\item {} 
値段

\item {} 
売り切れフラグ

\end{itemize}


\section{説明}
\label{\detokenize{user/index:id4}}\begin{itemize}
\item {} 
ユーザーが商品を選ぶ際の一覧

\end{itemize}


\subsection{補足}
\label{\detokenize{user/index:id5}}\begin{itemize}
\item {} 
売切商品も取得する(フラグで判別できます)。

\item {} 
業者側で登録を削除した商品は取得しない。

\item {} 
業者の在庫確認とは違い、登録されている商品は全て取得する。

\item {} 
在庫は取得しない。

\end{itemize}
\clearpage

\subsection{シーケンス図}
\label{\detokenize{user/index:id6}}
\noindent\sphinxincludegraphics{{index}.png}


\chapter{商品購入API}
\label{\detokenize{user/buy:api}}\label{\detokenize{user/buy::doc}}

\section{内容}
\label{\detokenize{user/buy:id1}}\begin{enumerate}
\item {} 
指定された商品を検索

\item {} 
料金が足りないかを判定する
\begin{itemize}
\item {} 
料金が足りたら
\begin{enumerate}
\item {} 
精算する

\item {} 
売上のログデータを生成する

\item {} 
該当商品の在庫を-1する

\end{enumerate}

\item {} 
料金が足りなかったら
\begin{enumerate}
\item {} 
エラーメッセージを返す

\end{enumerate}

\end{itemize}

\end{enumerate}


\subsection{リクエスト}
\label{\detokenize{user/buy:id2}}
/item/:id/buy

params = payment


\subsection{返ってくる値}
\label{\detokenize{user/buy:id3}}\begin{itemize}
\item {} 
商品名

\item {} 
購入時間

\item {} 
投入料金-商品の値段(お釣り)

\end{itemize}


\section{説明}
\label{\detokenize{user/buy:id4}}\begin{itemize}
\item {} 
ユーザーが商品購入の際の処理

\end{itemize}


\subsection{補足}
\label{\detokenize{user/buy:id5}}\begin{itemize}
\item {} 
インプットパラメータとしてURLに「payment」が必須です。(例:\textasciitilde{}\textasciitilde{}/item/1/buy?payment=130)

\item {} 
「payment」が入っていない、または数字ではない場合、エラーメッセージを返します。

\end{itemize}


\subsection{シーケンス図}
\label{\detokenize{user/buy:id6}}
\noindent\sphinxincludegraphics{{buy}.png}

\begin{DUlineblock}{0em}
\item[] 
\end{DUlineblock}


\chapter{商品の入庫}
\label{\detokenize{trader/stock::doc}}\label{\detokenize{trader/stock:id1}}

\section{内容}
\label{\detokenize{trader/stock:id2}}

\subsection{指定した商品を購入する}
\label{\detokenize{trader/stock:id3}}

\chapter{商品の新規作成}
\label{\detokenize{trader/create::doc}}\label{\detokenize{trader/create:id1}}

\section{内容}
\label{\detokenize{trader/create:id2}}

\subsection{説明}
\label{\detokenize{trader/create:id3}}

\chapter{商品の編集}
\label{\detokenize{trader/edit::doc}}\label{\detokenize{trader/edit:id1}}

\section{内容}
\label{\detokenize{trader/edit:id2}}

\subsection{説明}
\label{\detokenize{trader/edit:id3}}

\chapter{商品の削除}
\label{\detokenize{trader/destroy::doc}}\label{\detokenize{trader/destroy:id1}}

\section{内容}
\label{\detokenize{trader/destroy:id2}}

\subsection{説明}
\label{\detokenize{trader/destroy:id3}}

\chapter{商品の在庫確認}
\label{\detokenize{trader/index::doc}}\label{\detokenize{trader/index:id1}}

\section{内容}
\label{\detokenize{trader/index:id2}}

\subsection{説明}
\label{\detokenize{trader/index:id3}}

\chapter{商品の月次売上}
\label{\detokenize{trader/monthly_sales::doc}}\label{\detokenize{trader/monthly_sales:id1}}

\section{内容}
\label{\detokenize{trader/monthly_sales:id2}}

\subsection{説明}
\label{\detokenize{trader/monthly_sales:id3}}

\chapter{商品の個別売上}
\label{\detokenize{trader/indivisual_sales::doc}}\label{\detokenize{trader/indivisual_sales:id1}}

\section{内容}
\label{\detokenize{trader/indivisual_sales:id2}}

\subsection{説明}
\label{\detokenize{trader/indivisual_sales:id3}}
\begin{DUlineblock}{0em}
\item[] 
\end{DUlineblock}


\chapter{表(テーブル)の書き方}
\label{\detokenize{table::doc}}\label{\detokenize{table:id1}}
\noindent\begin{tabulary}{\linewidth}{|L|L|}
\hline
\multicolumn{2}{|l|}{\relax \sphinxstylethead{\relax 
栃木県内の勉強会
\unskip}\relax \unskip}\relax \\
\hline\multirow{2}{*}{\relax 
宇都宮
\unskip}\relax &
集合知勉強会
\\
\cline{2-2}&
Objective-C
\\
\hline
西那須野
&
とちぎRuby
\\
\hline\end{tabulary}


\noindent\begin{tabulary}{\linewidth}{|L|L|}
\hline
\multicolumn{2}{|l|}{\relax \sphinxstylethead{\relax 
勉強会で使う本
\unskip}\relax \unskip}\relax \\
\hline\sphinxstylethead{\relax 
言語
\unskip}\relax &\sphinxstylethead{\relax 
本の名前
\unskip}\relax \\
\hline
Ruby
&
dRubyによる分散・Webプログラミング
\\
\hline
Python
&
集合知プログラミング
\\
\hline
Objective-C
&
詳解Objective-C 2.0
\\
\hline\end{tabulary}



\begin{threeparttable}
\capstart\caption{Frozen Delights!}\label{\detokenize{table:id2}}
\noindent\begin{tabulary}{\linewidth}{|L|L|L|}
\hline
\sphinxstylethead{\relax 
1
\unskip}\relax &\sphinxstylethead{\relax 
2
\unskip}\relax &\sphinxstylethead{\relax 
3
\unskip}\relax \\
\hline
a
&
b
&
c
\\
\hline
c
&
d
&
f
\\
\hline
ssssssssssssssssssss
&
d
&
ggg
\\
\hline\end{tabulary}

\end{threeparttable}



\chapter{Indices and tables}
\label{\detokenize{index:indices-and-tables}}\begin{itemize}
\item {} 
\DUrole{xref,std,std-ref}{genindex}

\item {} 
\DUrole{xref,std,std-ref}{modindex}

\item {} 
\DUrole{xref,std,std-ref}{search}

\end{itemize}



\renewcommand{\indexname}{索引}
\printindex
\end{document}